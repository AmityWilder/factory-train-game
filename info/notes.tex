%%%%%%%%%%%%%%%%%%%%%%%%%%%%% Define Article %%%%%%%%%%%%%%%%%%%%%%%%%%%%%%%%%%
\documentclass{article}
%%%%%%%%%%%%%%%%%%%%%%%%%%%%%%%%%%%%%%%%%%%%%%%%%%%%%%%%%%%%%%%%%%%%%%%%%%%%%%%

%%%%%%%%%%%%%%%%%%%%%%%%%%%%% Using Packages %%%%%%%%%%%%%%%%%%%%%%%%%%%%%%%%%%
\usepackage{geometry}
\usepackage{graphicx}
\usepackage{amssymb}
\usepackage{amsmath}
\usepackage{amsthm}
\usepackage{empheq}
\usepackage{mdframed}
\usepackage{booktabs}
\usepackage{lipsum}
\usepackage{graphicx}
\usepackage{color}
\usepackage{psfrag}
\usepackage{pgfplots}
\usepackage{bm}
\usepackage{chemfig}
\usepackage[version=4]{mhchem}
\usepackage{chemformula}
\usepackage{elements}
%%%%%%%%%%%%%%%%%%%%%%%%%%%%%%%%%%%%%%%%%%%%%%%%%%%%%%%%%%%%%%%%%%%%%%%%%%%%%%%

%%%%%%%%%%%%%%%%%%%%%%%%%% Page Setting %%%%%%%%%%%%%%%%%%%%%%%%%%%%%%%%%%%%%%%
\geometry{a4paper}

%%%%%%%%%%%%%%%%%%%%%%%%%% Define some useful colors %%%%%%%%%%%%%%%%%%%%%%%%%%
\definecolor{ocre}{RGB}{243,102,25}
\definecolor{mygray}{RGB}{243,243,244}
\definecolor{deepGreen}{RGB}{26,111,0}
\definecolor{shallowGreen}{RGB}{235,255,255}
\definecolor{deepBlue}{RGB}{61,124,222}
\definecolor{shallowBlue}{RGB}{235,249,255}
%%%%%%%%%%%%%%%%%%%%%%%%%%%%%%%%%%%%%%%%%%%%%%%%%%%%%%%%%%%%%%%%%%%%%%%%%%%%%%%

%%%%%%%%%%%%%%%%%%%%%%%%%% Define an orangebox command %%%%%%%%%%%%%%%%%%%%%%%%
\newcommand\orangebox[1]{\fcolorbox{ocre}{mygray}{\hspace{1em}#1\hspace{1em}}}
%%%%%%%%%%%%%%%%%%%%%%%%%%%%%%%%%%%%%%%%%%%%%%%%%%%%%%%%%%%%%%%%%%%%%%%%%%%%%%%

%%%%%%%%%%%%%%%%%%%%%%%%%%%% English Environments %%%%%%%%%%%%%%%%%%%%%%%%%%%%%
\newtheoremstyle{mytheoremstyle}{3pt}{3pt}{\normalfont}{0cm}{\rmfamily\bfseries}{}{1em}{{\color{black}\thmname{#1}~\thmnumber{#2}}\thmnote{\,--\,#3}}
\newtheoremstyle{myproblemstyle}{3pt}{3pt}{\normalfont}{0cm}{\rmfamily\bfseries}{}{1em}{{\color{black}\thmname{#1}~\thmnumber{#2}}\thmnote{\,--\,#3}}
\theoremstyle{mytheoremstyle}
\newmdtheoremenv[linewidth=1pt,backgroundcolor=shallowGreen,linecolor=deepGreen,leftmargin=0pt,innerleftmargin=20pt,innerrightmargin=20pt,]{theorem}{Theorem}[section]
\theoremstyle{mytheoremstyle}
\newmdtheoremenv[linewidth=1pt,backgroundcolor=shallowBlue,linecolor=deepBlue,leftmargin=0pt,innerleftmargin=20pt,innerrightmargin=20pt,]{definition}{Definition}[section]
\theoremstyle{myproblemstyle}
\newmdtheoremenv[linecolor=black,leftmargin=0pt,innerleftmargin=10pt,innerrightmargin=10pt,]{problem}{Problem}[section]
%%%%%%%%%%%%%%%%%%%%%%%%%%%%%%%%%%%%%%%%%%%%%%%%%%%%%%%%%%%%%%%%%%%%%%%%%%%%%%%

%%%%%%%%%%%%%%%%%%%%%%%%%%%%%%% Plotting Settings %%%%%%%%%%%%%%%%%%%%%%%%%%%%%
\usepgfplotslibrary{colorbrewer}
\pgfplotsset{width=8cm,compat=1.9}
%%%%%%%%%%%%%%%%%%%%%%%%%%%%%%%%%%%%%%%%%%%%%%%%%%%%%%%%%%%%%%%%%%%%%%%%%%%%%%%

%%%%%%%%%%%%%%%%%%%%%%%%%%%%%%% Title & Author %%%%%%%%%%%%%%%%%%%%%%%%%%%%%%%%
\title{Chem Notes}
\author{Amy Wilder}
%%%%%%%%%%%%%%%%%%%%%%%%%%%%%%%%%%%%%%%%%%%%%%%%%%%%%%%%%%%%%%%%%%%%%%%%%%%%%%%

\begin{document}
\maketitle

\noindent
\ch{"\chlewis{0.}{H}"}\quad
\ch{"\chlewis{0:}{He}"}\\
\ch{"\chlewis{0.}{Li}"}\quad
\ch{"\chlewis{0.180.}{Be}"}\quad
\ch{"\chlewis{0.90.180.}{B}"}\quad
\ch{"\chlewis{0.90.180.270.}{C}"}\quad
\ch{"\chlewis{0.90:180.270.}{N}"}\quad
\ch{"\chlewis{0.90:180:270.}{O}"}\quad
\ch{"\chlewis{0.90:180:270:}{F}"}\quad
\ch{"\chlewis{0:90:180:270:}{Ne}"}\quad \\

\chemfig{H-[1]O-[7]H} \\

\chemfig{*6(-=(-CH_3)---(=G)=)}

\section{Solubility Rules for Ionic Compounds in Water}

\begin{tabular}{lp{4in}}
    \textbf{Ions} & \textbf{Comments} \\
    \hline
    \(\mathsf{{NH_4}^+, {Na}^+, {K}^+, {Li}^+}\)         & All common salts of these ions are soluble \\
    \(\mathsf{{NO_3}^-, {CH_3COO}^-}\)                   & All common salts of these ions are soluble \\
    \(\mathsf{{Cl}^-, {Br}^-, {I}^-}\)                   & All common salts of these ions are soluble, \textbf{\_except\_} those of \(\mathsf{{Ag}^+, {Pb}^{2+}, {Cu}^+,}\) and \(\mathsf{{{Hg}_2}^+}\). \\
    \(\mathsf{{SO_4}^{2-}}\)                             & Most are soluble \textbf{\_except\_} those containing \(\mathsf{{Ca}^{2+}, {Sr}^{2+}, {Ba}^{2+},}\) and \(\mathsf{{Pb}^{2+}}\) ions \\
    \(\mathsf{{OH}^-}\)                                  & All are insoluble, \textbf{\_except\_} those with \(\mathsf{{NH_4}^+}\) ion, Group 1A, and the larger members of Group 2A beginning with \(\mathsf{{Ca}^{2+}}\) \\
    \(\mathsf{{CO_3}^{2-}, {PO_4}^{3-}, {C_2O_4}^{2-}}\) & All common salts are insoluble, \textbf{\_except\_} those of Group 1A and \(\mathsf{{NH_4}^+}\) \\
\end{tabular}

\section{Common ions and their respective charges}

Ion Summary: When cations pair up with anions to form neutral ionic compounds, the word ``ion'' is dropped from both ion's names

\textbf{Main group metals an Nonmetals:} predictable charges based upon their group/column.

\begin{tabular}{ll}
    \textbf{Name} & \textbf{Cation} \\
    \hline
    Lithium ion   & \(\mathsf {Li}^{1+}\) \\
    Sodium ion    & \(\mathsf {Na}^{1+}\) \\
    Potassium ion & \(\mathsf K^{1+}\)    \\
    Rubidium ion  & \(\mathsf {Rb}^{1+}\) \\
    Cesium ion    & \(\mathsf {Cs}^{1+}\) \\
    ~             &
\end{tabular}
\begin{tabular}{ll}
    \textbf{Name} & \textbf{Cation} \\
    \hline
    Berylium ion  & \(\mathsf {B}e^{2+}\) \\
    Magnesium ion & \(\mathsf {Mg}^{2+}\) \\
    Calcium ion   & \(\mathsf {Ca}^{2+}\) \\
    Strontium ion & \(\mathsf {Sr}^{2+}\) \\
    Barium ion    & \(\mathsf {Ba}^{2+}\) \\
    Aluminum ion  & \(\mathsf {Al}^{3+}\) \\
\end{tabular}

\begin{tabular}{ll}
    \textbf{Name} & \textbf{Anion} \\
    \hline
    Oxide ion     & \(\mathsf O^{2-}\) \\
    Sulfide ion   & \(\mathsf S^{2-}\) \\
    ~             &              \\
    Nitride ion   & \(\mathsf N^{3-}\) \\
    Phosphide ion & \(\mathsf P^{3-}\) \\
    ~             &              \\
\end{tabular}
\begin{tabular}{ll}
    \textbf{Name} & \textbf{Anion} \\
    \hline
    Bromide ion  & \(\mathsf {Br}^{1-}\) \\
    Chloride ion & \(\mathsf {Cl}^{1-}\) \\
    Fluoride ion & \(\mathsf F^{1-}\)    \\
    Iodine ion   & \(\mathsf I^{1-}\)    \\
    ~            &                 \\
    ~            &                 \\
\end{tabular}

\textbf{Transition Metals} (and other metals with d orbitals): The charge for many of these can vary. Therefore Roman Numerals are used in their written name to indicate the exact ion. If a transition metal is in the cmpd, then you need to look at the anion to determine the charge of the metal cation.

\begin{tabular}{ll}
    \textbf{Name} & \textbf{Cation} \\
    \hline
    Cadmium ion & \(\mathsf {Cd}^{2+}\) \\
    Zinc ion    & \(\mathsf {Zn}^{2+}\) \\
    Silver ion  & \(\mathsf {Ag}^{2+}\) \\
    Gold ion    & \(\mathsf {Au}^{2+}\) \\
    ~           &                 \\
\end{tabular}
\begin{tabular}{ll}
    \textbf{Name} & \textbf{Cation} \\
    \hline
    Cobalt \(\rm(II)\) ion    & \(\mathsf {Co}^{2+}\) \\
    Cobalt \(\rm(III)\) ion   & \(\mathsf {Co}^{3+}\) \\
    Iron \(\rm(II)\) ion      & \(\mathsf {Fe}^{2+}\) \\
    Iron \(\rm(III)\) ion     & \(\mathsf {Fe}^{3+}\) \\
    Chromium \(\rm(III)\) ion & \(\mathsf {Cr}^{3+}\) \\
\end{tabular}


\begin{tabular}{ll}
    \textbf{Name} & \textbf{Cation} \\
    \hline
    Copper \(\rm(I)\) ion     & \(\mathsf {Cu}^{1+}\) \\
    Copper \(\rm(II)\) ion    & \(\mathsf {Cu}^{2+}\) \\
    Mercury \(\rm(II)\) ion   & \(\mathsf {Hg}^{2+}\) \\
    Manganese \(\rm(II)\) ion & \(\mathsf {Mn}^{2+}\) \\
    ~                       &                 \\
\end{tabular}
\begin{tabular}{ll}
    \textbf{Name} & \textbf{Cation} \\
    \hline
    Tin \(\rm(II)\) ion    & \(\mathsf {Sn}^{2+}\) \\
    Tin \(\rm(IV)\) ion    & \(\mathsf {Sn}^{4+}\) \\
    Lead \(\rm(II)\) ion   & \(\mathsf {Pb}^{2+}\) \\
    Lead \(\rm(IV)\) ion   & \(\mathsf {Pb}^{4+}\) \\
    Nickel \(\rm(II)\) ion & \(\mathsf {Ni}^{2+}\) \\
\end{tabular}

\textbf{POLYATOMIC ions.} These ions contain covalently bonded atoms with an overall charge. They remain together in a GROUP.

\begin{tabular}{ll}
    \textbf{Name} & \textbf{Cation} \\
    \hline
    Ammonium ion & \(\mathsf {NH_4}^{1+}\) \\
    ~            &                   \\
    ~            &                   \\
\end{tabular}
\begin{tabular}{ll}
    \textbf{Name} & \textbf{Anion} \\
    \hline
    Nitrate ion   & \(\mathsf {NH_4}^{1-}\) \\
    Hydroxide ion & \(\mathsf {OH_4}^{1-}\) \\
    Cyanide ion   & \(\mathsf {CN_4}^{1-}\) \\
\end{tabular}

\begin{tabular}{ll}
    \textbf{Name} & \textbf{Anion} \\
    \hline
    Sulfate ion   & \(\mathsf {SO_4}^{2-}\)  \\
    Chromate ion  & \(\mathsf {CrO_4}^{2-}\) \\
    Carbonate ion & \(\mathsf {CO_3}^{2-}\)  \\
\end{tabular}
\begin{tabular}{ll}
    \textbf{Name} & \textbf{Anion} \\
    \hline
    Phosphate ion & \(\mathsf {PO_4}^{3-}\)      \\
    ~             &                        \\
    Acetate ion   & \(\mathsf {C_2H_3O_2}^{1-}\) \\
\end{tabular}

\textbf{Strong Acids are Molecular Compounds} (typically the formula starts with \textbf{``H''} followed by an anion).

\begin{itemize}
    \item \(\mathsf{HCl}\) = hydrochloric acid
    \item \(\mathsf{HBr}\) = hydrobromic acid
    \item \(\mathsf{H_2SO_4}\) = sulfuric acid
    \item \(\mathsf{HNO_3}\) = nitric acid
\end{itemize}

More bonds require more energy to break. Shorter bonds require more energy to break (i.e. smaller atoms make stronger bonds).

\section{Geometry}

sp = linear, sp2 = trigonal planar, sp3 = quadragonal

\section{Intermolecular Forces}

IMFs

\begin{tabular}{p{1in}p{1in}p{1in}p{1in}}


\end{tabular}

\section{Dynamic Equilibrium}

When two opposite processes reach the same rate so that there is no gain or loss of material.
\begin{itemize}
    \item This \textbf{does not} mean there are equal amounts of vapor and liquid; it means that they are \textit{\textbf{changing}} by equal amounts.
    \item
    \item
\end{itemize}

The pressure exerted by vapor when it is in dynamic equilibrium with its liquid is called the \textbf{vapor pressure}.

The weaker the attractive forces between the molecules, the more molecules will be in the vapor. Therefore,

\begin{itemize}
    \item The weaker the attractive forces, the higher the vapor pressure.
    \item The higher the vapor pressure, the more volatile the liquid.
\end{itemize}

Dynamic equilibrium: Rate of vaporization = rate of condensation

Volume is increased, pressure falls. More gas vaporizes. Pressure is restored.

Volume is decreased, pressure rises. More gas condenses, pressure is restored.

\subsection{Vapor pressure vs temperature}

\[760\mathrm{torr} = 1\mathrm{atm} = 700\mathrm{mmHg}\]

\begin{itemize}
    \item Increasing temperature increases the number of molecules able to escape the liquid.
    \item The net result is that as temperature increases, the vapor pressure increases
    \item Small changes in temperature can make big changes in vapor pressure.
    \begin{itemize}
        \item  The rate of growth depends omn the strnegth of the intermolecular forces.
        \item vapor pressure vs temperature curves graphically represent the relationship
    \end{itemize}
\end{itemize}

\end{document}